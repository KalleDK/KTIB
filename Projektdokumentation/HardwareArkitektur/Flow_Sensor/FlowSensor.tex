%!TEX root = ../../main.tex

\section{Flowsensor}
Til kontrol at vandflow til system er valgt at benytte en flow sensor af typen YF S201. 
Dette er en Hall Effect sensor. En Hall Effekt sensor benytter ændringer i et nær-magnetfelt til at 
ændre sensorens outputspænding, dette genererer et PWM-output som identificerer den mængde vand 
der flyder igennem sensoren pr. tidsenhed.

Flowraten udregnes efter formlen: 
				
\begin{figure}[!h]
    \begin{align*}
       \frac{Pulse frequency [Hz]}{7.5} = flowrate[L/min]
    \end{align*}
\label{eq:PWM}
\caption{Beregning af flowrate}
\end{figure}				

Ved denne udregning svarer eks. 16Hz = 2L/min, og 65.5HZ = 8L/min. 
Flowsensoren tilkobles via 3 medfølgende pins.

\begin{figure}[!h]
	\begin{center}
		\begin{tabular}{ l l }
			 \textcolor{black}{Sort}:   & $GND(-)$ 		\\ 
			 \textcolor{yellow}{Gul}:   & $PWM output$ 	\\  
			 \textcolor{red}{Red}:    	& $VCC(+5V)$ 	\\
		\end{tabular}
	\end{center}
\caption{Værdier hentet fra datablad}
\end{figure}

Output fra sensoren følger TTL-standarden, det vil sige at den kan kobles direkte til en given inputpin på PSoC'en. Output-raten gives som udgangspunkt med en målepræcision på +/- 10%.
 
Under operation sinker sensoren 15mA, derved kan den kan trække sin forsyning direkte fra PSoC'en.

