%!TEX root = ../../main.tex

\subsection{Overordnet}
Til kommunikation i AVS har vi valgt at implementere en protokol der bruges til kommunikation mellem hardware enheder.
Det drejer sig om to busser der følger samme format, samt er der gjort brug af I2C i forhold til følere. De to busser Kar og Ø bus basere sig begge på samme data format da deres fysiske lag er ens. Busserne kører master/slave kommunikation det gør at det kun er master der kan initiere kommunikation.

\subsection{Dataformat - Kar og Ø bus} 
Data bliver sendt på følgende format:

\begin{enumerate}
\item Adressen på modtager af besked
\item Adressen på afsender af besked
\item Længden af de kommende argumenter (kan være 0)
\item Kommando der ønskes behandlet
\item Argumenter i henhold til længden
\end{enumerate}

Systemet er inddelt i forespørgsler og svar, grundet master/slave formatet. Derfor kan en hel besked betragtes som en forespørgsel fra master og et svar fra slave.

Forespørgsler og svar har samme format som beskrevet ovenfor.

En kommunikation mellem master og slave kunne se således ud:
Master vil sende på dette format:
\begin{table}[H]
\setlength{\parindent}{12pt}
\begin{tabular}{|l|l|c|c|c|}\hline
\multicolumn{5}{|l|}{Master\cellcolor[gray]{0.9}}\\\hline
RxAddr & TxAddr & len & cmd & args \\\hline
1 Byte & 1 Byte & 1 Byte & 1 Byte & len * 1 Bytes \\\hline 
\end{tabular}
\end{table}


Slave vil svare på dette format:
\begin{table}[H]
\setlength{\parindent}{12pt}
\begin{tabular}{|l|l|c|c|c|}\hline
\multicolumn{5}{|l|}{Slave\cellcolor[gray]{0.7}}\\\hline
RxAddr & TxAddr & len & cmd & args  \\\hline
1 Byte & 1 Byte & 1 Byte & 1 Byte & len * 1 bytes \\\hline 
\end{tabular}
\end{table}

Alt efter hvilken kommando der sendes vil det variere om der er argumenter med det vil sige at len godt kan være 0. Som det også kan ses er master/slave forespørgsel og svar ens i opbygningen.

\subsection{Kar bus kommandoer}
Alle kommandoer er delt op på et "request" / "response" format dette er pga. af master/slave kommunikationen, det gør så at master altid kan forvente at få et svare og hvis ikke der bliver svaret er der sket en fejl.

Der er implementere følgende kommandoer på KarBus:

\begin{table}[H]
\setlength{\parindent}{12pt}
\begin{tabular}{|l|l|}\hline
\multicolumn{2}{|c|}{KarBus\cellcolor[gray]{0.7}}\\\hline
Kommando & Beskrivelse \\\hline
REQ\_KAR\_SENSOR\_DATA 		& Forespørgsel på at få data fra sensor der er koblet til kar \\\hline 
RES\_KAR\_SENSOR\_DATA 		& Svar indeholdende sensor data								 \\\hline 
REQ\_KAR\_VENTIL	   		& Forespørgsel på at skifte tilstand på ventil der er koblet til kar \\\hline 
RES\_KAR\_VENTIL       		& Svar der indeholder den tilstand ventilen er sat til \\\hline 
REQ\_KAR\_PUMPE		   		& Forespørgsel på at skifte hastighed på pumpen \\\hline 
RES\_KAR\_PUMPE 	   		& Svar indeholdende den hastighed pumpen der sat til \\\hline 
REQ\_KAR\_OPRET 	   		& Forespørgsel om at oprette en ny sensor ø \\\hline
RES\_KAR\_OPRET 	   		& Svar med adressen på den nye ø \\\hline 
REQ\_KAR\_OE\_LIST	   		& Forespørgsel om at oprette en ny sensor ø \\\hline
RES\_KAR\_OE\_LIST	   		& Svar med adressen på den nye ø \\\hline 
REQ\_KAR\_OE\_SENSOR\_DATA	& Forespørgsel på at få sensors øens sensor data \\\hline
RES\_KAR\_OE\_SENSOR\_DATA 	& Svar med data for sensor ø og om sensor er koblet til eller ej \\\hline 
REQ\_KAR\_OE\_VENTIL 	    & Forespørgsel om at åbne/lukke sensor øens ventil \\\hline
RES\_KAR\_OE\_VENTIL 	    & Svar med den tilstand ventilen er efterlad i \\\hline
REQ\_KAR\_OE\_SENSOR\_TYPE 	& Forespørgsel om hvilken type føler der er tilsluttet øen \\\hline
RES\_KAR\_OE\_SENSOR\_TYPE 	& Svar med sensor type \\\hline  
\end{tabular}
\end{table}

\subsubsection{REQ\_KAR\_SENSOR\_DATA}
Denne kommando er en forespørgsel på data som Karrets sensorer har indsamlet.

\begin{table}[H]
\setlength{\parindent}{12pt}
\begin{tabular}{|l|lcc|}
Kommando & 0x1 & & \\\hline
Argumenter & ingen & & \\
\end{tabular}
\end{table}



\subsubsection{RES\_KAR\_SENSOR\_DATA}
Denne kommando er et svar der indeholder den data karrets sensorer har indsamlet.

\begin{table}[H]
\setlength{\parindent}{12pt}
\begin{tabular}{|l|lcc|}
Kommando & 0x2 & & \\
Argumenter & ID & Value1 & Value2 \\
\end{tabular}
\end{table}

Argumenterne kan forekomme flere gange og vil være reflekteret i længden. Det vil sige at en længde på 3 i dette tilfælde betyder at der kun er indeholdt en sensor i svaret, hvis længden var 6 ville der være to sensorer osv.

\begin{table}[H]
\setlength{\parindent}{12pt}
\begin{tabular}{|l|l|}
ID & Dette er følerens identifikation alt efter om det er pH, flow eller andet der bliver målt. \\
Value1 & Dette er første del af den målte værdi. \\
Value2 & Dette er anden del af den målte værdi.\\
\end{tabular}
\end{table}


\subsubsection{REQ\_KAR\_VENTIL}
Denne kommando er en forespørgsel på at styre karrets ventiler.

\begin{table}[H]
\setlength{\parindent}{12pt}
\begin{tabular}{|l|lcc|}
Kommando & 0x3 & & \\
Argumenter & ID & STATE & \\
\end{tabular}
\end{table}

Her er ID den ventil der ønske manipuleret og STATE er hvilken tilstand der ønskes at ventilen skal stå i.\\\\
STATE kan være følgende:
\begin{table}[H]
\setlength{\parindent}{12pt}
\begin{tabular}{|l|l|}
Lukket & 0x0 \\
Åben & 0x1 \\
\end{tabular}
\end{table}

\subsubsection{RES\_KAR\_VENTIL}
Denne kommando er et svar der indeholder den tilstand ventilen er i.

\begin{table}[H]
\setlength{\parindent}{12pt}
\begin{tabular}{|l|lcc|}
Kommando & 0x4 & & \\
Argumenter & STATE & & \\
\end{tabular}
\end{table}

Svaret på en ventil forespørgsel er hvilken tilstand ventilen er blevet sat til.

\subsubsection{REQ\_KAR\_PUMPE}
Denne kommando er en forespørgsel på at styre karrets pumpe.

\begin{table}[H]
\setlength{\parindent}{12pt}
\begin{tabular}{|l|lcc|}
Kommando & 0x5 & & \\
Argumenter & STATE & & \\
\end{tabular}
\end{table}


Da pumpen kan køre med flere hastigheder sendes der en state der kan være følgende:

\begin{table}[H]
\setlength{\parindent}{12pt}
\begin{tabular}{|l|l|}
Slukket & 0 \\
Lav hastighed & 25 \\
Middel hastighed & 50 \\
Høj hastighed & 75 \\
\end{tabular}
\end{table}


\subsubsection{RES\_KAR\_PUMPE}
Denne kommando er et svar der indeholder den tilstand pumpen er i.

\begin{table}[H]
\setlength{\parindent}{12pt}
\begin{tabular}{|l|lcc|}
Kommando & 0x6 & & \\
Argumenter & STATE & & \\
\end{tabular}
\end{table}


Svaret på en pumpe forespørgsel er hvilken tilstand pumpen er blevet sat til.


\subsubsection{REQ\_KAR\_OPRET}
Denne kommando er en forespørgsel give et kar en adresse.

\begin{table}[H]
\setlength{\parindent}{12pt}
\begin{tabular}{|l|lcc|}
Kommando & 0x7 & & \\
Argumenter & ADDR & & \\
\end{tabular}
\end{table}


\subsubsection{RES\_KAR\_OPRET}
Denne kommando er et svar der fortæller om det lykkedes at oprette karret.

\begin{table}[H]
\setlength{\parindent}{12pt}
\begin{tabular}{|l|lcc|}
Kommando & 0x8 & & \\
Argumenter & ADDR & & \\
\end{tabular}
\end{table}


\subsubsection{REQ\_KAR\_OE\_LIST}
Denne kommando er en forespørgsel give et kar en adresse.

\begin{table}[H]
\setlength{\parindent}{12pt}
\begin{tabular}{|l|lcc|}
Kommando & 0x9 & & \\
Argumenter & ingen & & \\
\end{tabular}
\end{table}


\subsubsection{RES\_KAR\_OE\_LIST}
Denne kommando er et svar der fortæller om det lykkedes at oprette karret.

\begin{table}[H]
\setlength{\parindent}{12pt}
\begin{tabular}{|l|lcc|}
Kommando & 0x10 & & \\
Argumenter & ingen & & \\
\end{tabular}
\end{table}