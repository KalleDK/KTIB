%!TEX root = ../../../main.tex
\subsection{Doseringspumpe}
Følgende krav er opstillet for doseringspumpen.  

\begin{table}[h]
	\begin{tabular}{ l l l }
		1. 	& Tolerance for Vandtryk:   	& Skal kunne klare min. 2 bar \\
		2. 	& Forsyningsspænding: 			& Skal kunne benytte 12V DC \\
		3.	& Tolerance for Temp: 			& Skal kunne operere ved 45 C \\
		4.	& Flow-regulering: 				& Skal kunne levere min. 0.5 liter/min
	\end{tabular}
\end{table}

Her bør opmærksomheden primært henledes på vandtrykket der kan opstå i slangen, imellem doseringspumpen og doseringsventilerne, hvis ingen af disse ikke er åbne når pumpen tændes.
Dette problem vil praktisk blive løst software-wise, således at det ikke er muligt at tænde for pumpen med mindre at min. Én af de tilkoblede ventiler er åbne.  

Der blev undersøgt flere typer pumper i forhold til de ovenstående krav, og det blev besluttet at benytte en inline pumpe til opgaven. Dette blev primært besluttet på baggrund af dens performance og forsyning, denne passer til designet at systemet.  
Den valgte model blev: 
Biltema inline Pump, 17l/min, DC12V, 3A, 
Datablad for den findes under Datablade. 
 