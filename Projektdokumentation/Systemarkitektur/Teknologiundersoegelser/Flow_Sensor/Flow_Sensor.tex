%!TEX root = ../../../main.tex
\subsection{Flowsensor}
Følgende krav er opstillet for flowsensoren 

\begin{table}[h]
	\begin{tabular}{ l l l }
		1. 	& Tolerance for Vandtryk:   	& Skal kunne klare min. 2 bar \\
		2. 	& Forsyningsspænding: 			& Skal kunne benytte 12V DC \\
	\end{tabular}
\end{table}

Det vigtigste krav til flowsensoren er, at den bør være rated til at kunne klare det tryk, der påhviler den. 
For flowsensoren gælder følgende grænseflader: 

\begin{table}[h]
	\begin{tabular}{ l l l }
		1. 	& Vandtilførsel udefra   		& \\
		2. 	& Vandkaret: 					&
	\end{tabular}
\end{table}

Her bør opmærksomheden primært henledes på vandtrykket udefra. 
Der tages udgangspunkt i den alm. Vandhane, her i er vandtrykket som standard på omkring 2 bar . 
Derfor skal den ventil der vælges som min. kunne klare et tryk 2 bar. 
For afløbsventilen gælder følgende grænseflade: 

\begin{table}[h]
	\begin{tabular}{ l l l }
		1. 	& Vandkaret:  					& \\
		2. 	& Afløb: 						&
	\end{tabular}
\end{table}

Her bør opmærksomheden primært henledes på vandtrykket udefra. 
Der tages udgangspunkt i den alm. Vandhane, her i er vandtrykket som standard på omkring 2 bar . 
Derfor skal den flowsensor der vælges som min. kunne klare et tryk 2 bar. 

Der blev undersøgt flere typer ventiler i forhold til de ovenstående krav, og det blev besluttet at benytte en YF-S201 Hall Effekt flow sesnor. Dette blev primært besluttet på baggrund af måden hvorpå den afgiver sine målinger, dette passer til designet at resten systemet.  
Den valgte model blev: 
Sea, 2 Porte, YF-S201 Hall Effekt flowsensor, 12v DC,  1/2tommer.
Datablad for den findes under Datablade. 