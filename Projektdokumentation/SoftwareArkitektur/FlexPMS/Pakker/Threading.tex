FlexPMS er dybt afhængig af trådteknologi. De tre store komponenter, Socket serveren, Bridge og Kar bus, kører parallelt i hver sin tråd. Trådene kommunikerer med hinanden gennem et event-baseret beskedsystem. Al trådhåndtering er skrevet specifikt til at køre på Linux, og FlexPMS er derfor ikke understøttet af andre operativsystemer.

\KlasseDiagram{0.5}{FlexPMS}{Thread}

\subsubsection{Thread}
\textit{Thread} er en abstrakt basis-klasse for alle klasser, som skal afvikles i sin egen tråd. Ved at nedarve fra \textit{Thread} kan en klasse nøjes med at implementere en \textKode{run()} metode, der kaldes, når tråden startes via \textKode{start()}. Tråden lever indtil \textKode{run()} returnerer, eller indtil der kaldes \textKode{cancel()} på en tråd, og tråden dør. \textit{Thread} er udelukkende skrevet til at understøtte pthread på Linux.

\StateDiagram{1}{FlexPMS}{Thread}

\subsubsection{Arkitekturspecifikke metoder}

\funk{void start()}{Starter tråden}{Ingenting}
{}

\funk{void cancel()}{Stopper tråden, hvis annullering er slået til, ellers gør funktionen ingenting. Tråden stoppes først når der stødes på et såkaldt cancellation point. Kun tråden selv kan tillade eller forbyde annullering}{Ingenting}
{}

\funk{void join()}{Blokerende kald, som ikke returnerer før tråden er færdig med eksekvering}{Ingenting}
{}

\funk{virtual void run()}{Abstrakt metode, som kaldes når tråden startes. Tråden lever så længe run() er under afvikling, eller indtil den annulleres}{Ingenting}
{}

\funk{static void* run\_thread(void* arg)}{C-style funktion hvori tråden startes. Denne funktion kaldes af \textKode{start()} og kalder til gengæld \textKode{run()} på Thread-objektet}{Ingenting}
{
\funkArg{arg}{En pointer til det Thread-objekt, som skal køres i en tråd}
}

\funk{void enable\_cancel()}{Tillader annullering af tråden, så tråden kan stoppes hvis \textKode{cancel()} kaldes}{Ingenting}
{}

\funk{void disable\_cancel()}{Forbyder annullering af tråden, så hvis \textKode{cancel()} kaldes ignoreres det}{Ingenting}
{}

\funk{void ssleep(unsigned int sec)}{Lægger tråden til at sove i et antal sekunder (minimum)}{Ingenting}
{
\funkArg{sec}{Antal sekunder tråden minimum skal sove i}
}

\funk{void msleep(unsigned int msec)}{Lægger tråden til at sove i et antal millisekunder (minimum))}{Ingenting}
{
\funkArg{msec}{Antal millisekunder tråden minimum skal sove i}
}
