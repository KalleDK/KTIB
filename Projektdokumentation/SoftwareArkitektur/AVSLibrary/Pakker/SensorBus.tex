\subsubsection{Formål}
SensorBussen er en overbygning på en I2C Multi-Master-Slave Bus. 
Der bør kun sidde en Sensor Ø på bussen, men op til 10 Fieldsensors 
(begrænset i Sensor Ø’ens hukommelse). Hovedformålet er at hive 
sensordata ud af Fieldsensorne, derfor er SensorPoll bygget op til 
kun at hive måleværdier ud som default. Hvis Fieldsensortypen er ukendt, 
kan denne også trække ud, bemærk at der bliver skiftet tilbage til Value, 
derved er der mindst muligt overhead i de fremtidige afmålinger. 
Der er startet op på en Scan funktion, samt en automatisk tildeling 
af adresser til Fieldsensor. Disse er ikke blevet færdigjort, derfor 
heller ikke dokumenteret i nedenstående. Scan benytter sig af 
non-blocking read, til at teste om en enhed svarer.

\SekvensDiagram{0.82}{AVSLibrary}{SensorPoll}

\subsubsection{Funktioner}

\funk{void SB\_Start(void)}
{Initialisere I2C komponenten, samt indlæser dummy værdier indtil reelle værdier er klar.}
{Void}
{}

\funk{void SB\_Communicate()}
{Er funktionen der kalder ParseRead og ParseWrite, denne funktion bør kaldes i mainloopet}
{Void}
{}

\funk{uint8 SB\_Read(uint8 addr, uint8 * str, uint8 len, uint8 blocking\_mode)}
{Der er to modes til denne kommando blocking og 
non-blocking. I blocking lader den hardwaren om al 
kommunikation, dog kan dette ende i deadlock. 
i non-blocking mode tester den om enheden eksisterer 
og kan kommunikeres med.}
{Success: 1 Failure: Anything else}
{
\funkArg{addr}{Adressen der skal læses fra}
\funkArg{str}{Char pointer hvor dataen fra læsningen gemmes}
\funkArg{len}{Hvor mange chars der skal læses}
\funkArg{blocking\_mode}{Sætter blocking mode}
}

\funk{uint8 SB\_Write(uint8 addr, uint8 * str, uint8 len, uint8 blocking\_mode)}
{Se SB\_Read}
{Success: 1 Failure: Anything else}
{
\funkArg{addr}{Adressen der skal skrives til}
\funkArg{str}{Char pointer hvor dataen fra skrivningen hentes}
\funkArg{len}{Hvor mange chars der skal skrives}
\funkArg{blocking\_mode}{Sætter blocking mode}
}

\funk{void SB\_ParseRead(void)}
{Nulstiller readpointeren efter læsning.}
{Void}
{}

\funk{void SB\_ParseWrite(void)}
{Fieldsensor: Skifter hvilken værdi der skal returneres, hvis Sensor Øen har bedt om dette. Sensor Ø: Bruges til at verificere at Fieldsensoren har skiftet mode}
{Void}
{}

\funk{void SB\_ChangeMode(uint8 mode)}
{Fieldsensor: Skifter mode mellem Value og Type}
{Void}
{
\funkArg{mode}{SB\_REQ\_TYPE, SB\_REQ\_VALUE}
}

\funk{void SB\_RefreshData()}
{Opdatere read bufferen med Type eller Value værdierne}
{Void}
{}

\funk{void SB\_LoadValue(uint8 high, uint8 low)}
{Bruges af Fieldsensoren til at opdatere værdien der måles}
{Void}
{
\funkArg{high}{0x00 - 0xFF Værdien har betydning efter hvilken Fieldsensor der anvendes}
\funkArg{low}{0x00 - 0xFF Værdien har betydning efter hvilken Fieldsensor der anvendes}
}

\funk{void SB\_SensorChangeMode(uint8 addr, uint8 mode)}
{Sensor Ø: Få en Fieldsensor til at skifte mode}
{Void}
{
\funkArg{addr}{Adressen på Fieldsensoren}
\funkArg{mode}{Hvilken mode der skal skiftes til}
}

\funk{void SB\_SensorPoll(uint8 addr, uint8 *values, uint8 *type)}
{Sensor Ø: Henter værdier samt type fra en Fieldsensor}
{Success: 1 - Failure: 0}
{
\funkArg{addr}{Adressen på Fieldsensoren}
\funkArg{values}{uint8 pointer hvor values gemmes i}
\funkArg{type}{uint8 pointer hvor typen gemmes i}
}

\funk{void SB\_DebugHandle(const char ch)}
{Kan bl.a. læse og skrive I2C direkte, læse i buffers, scanne efter enheder mm. Debug handler,
se mere under DebugUart}
{Void}
{
\funkArg{ch}{Input char}
}