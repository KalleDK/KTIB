\subsubsection{Formål}
Denne komponents formål er at overtage 
printf() funktionen, og sende data strømmen fra denne ud 
via en UART. Ideen er at programmet kan kompileres med 
og uden debug. Hvis debug er deaktiveret bliver debug 
data strømmen sendt i et sort hul, ellers bliver det 
sendt ud af en UART. Dette gør at andre komponenter 
blot skal inludere en header fil, og kan derefter bruge 
printf() uden at skulle tænke videre over det. Desuden giver 
komponenten også mulighed for andre komponenter at koble 
sig på en debug menu, så man kan sende kommandoer direkte 
til de individuelle komponenter. Dette er gjort via 
funktiones pointere. Når en anden komponent bliver tilføjet 
bliver denne automatisk givet et nummer i menuen. Når et 
nummer bliver indtastet skifter DebugHandle til denne 
komponent. Brugeren kan derefter bruge ESC for at komme 
tilbage til menuen og vælge en ny DebugHandle.

\StateDiagram{0.82}{AVSLibrary}{DebugMenu}

\subsubsection{Funktioner}
Her ses funktionerne der er tilknyttet komponenten.

\funk{int Debug\_printf(const char* string, ...)}
{Er dokumenteret her, da det er en fri funktion, men hører til denne komponent.}
{Success: Antal chars udskrevet Failure: Negative number}
{
\funkArg{string}{Nultermineret char array}
\funkArg{...}{Se printf dokumentation for variable argumenter}
}

\funk{void Debug\_Start(void)}
{Kobler UART'ens PutString til printf funktionen. Samt starter denne op.}
{Void}
{}

\funk{void Debug\_PutChar(char8 txDataByte)}
{Wrapper for UART\_PutChar}
{Void}
{
\funkArg{txDataByte}{Char der skal sendes via UART}
}

\funk{void Debug\_PutString(const char8 string[])}
{Wrapper for UART\_PutString}
{Void}
{
\funkArg{string}{Nultermineret char array med chars der skal sendes via UART}
}

\funk{void Debug\_Communicate()}
{Redirecter input chars fra UART til den valgte DebugHandle}
{Void}
{}

\funk{void Debug\_DebugHandle(const char ch)}
{Bruges til at skifte DebugHandle og udskrive en liste af Komponenter der er tilkoblet debug}
{Void}
{}

\funk{void Debug\_AddComponent(void (*debugchar) (char))}
{Tilføjer en DebugHandle funktion, bruges til at koble komponenters DebugHandle til.}
{Void}
{}
