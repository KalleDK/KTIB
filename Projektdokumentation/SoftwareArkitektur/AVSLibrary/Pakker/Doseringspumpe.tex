\subsubsection{Formål}
Formålet med denne komponent er at lave en ens tilgang 
til pumper i systemet. Komponenten består af et Clock 
input, en PWM generator, samt en Pin. Clocken er input 
til PWM generatoren, som outputter sit signal til Pin'en. 
Dette signal bliver derefter brugt på et pumpe 
styreboard. Selve PWM signalet er et Fast PWM signal 
hvor vi sætter comparatoren efter de procenter den 
bliver indstillet til. Der kan på komponenten vælges om 
signalet skal invertes. Hastigheden af PWM signalet 
angives i procent. Komponenten har 2 states running (0x01) 
og blocked (0x00). Dette angiver om PWM generatoren er 
indstillet til at sende det ønskede signal. 
Man kan derved instille en hastighed inden man starter 
pumpen, skulle dette være ønsket, samt huske pumpen den 
sidste værdi den var indstillet til, hvis man midlertidig 
vil stoppe pumpen og starte den igen med samme hastighed.

\subsubsection{Funktioner}
Her ses funktionerne der er tilknyttet komponenten.

\funk{void Pumpe\_Start(void)}
{Funktionen bruges til at initialisere PWM generatoren}
{Void}
{}

\funk{void Pumpe\_Run(void)}
{Indstiller PWM generatoren med den forprogrammerede hastighed, hvis den tidligere hastighed var 0\%, bliver hastigheden sat til 100\%. Pumpens state bliver ændret til running (0x01)}
{Void}
{}

\funk{void Pumpe\_Block(void)}
{Stopper PWN generatoren og ændre pumpens state til blocked (0x00)}
{Void}
{}

\funk{void Pumpe\_SetSpeed(uint8 percent)}
{Indstiller hastigheden PWM generatoren ønskes at køre med, er staten running (0x01) bliver PWM generatoren's comperator ændret med det samme}
{Void}
{
\funkArg{percent}{0 - 100, hvor mange procent af tiden signalet skal være aktivt.}
}

\funk{void Pumpe\_ApplySpeed(void)}
{Intern funktion der indstiller PWM Generatoren med den ønskede hastighed}
{Void}
{}

\funk{uint8 Pumpe\_GetSpeed(void)}
{Returnerer den indstillede hastighed i procent}
{0 - 100 procent}
{}

\funk{void Pumpe\_DebugHandle(const char ch)}
{Kan bl.a. starte, stoppe pumpen, samt ændre hastighed på signalet. Debug handler, se mere under DebugUart}
{Void}
{
\funkArg{ch}{Input char}
}

\funk{void Pumpe\_DebugState(void)}
{Udskriver hvilken state pumpen er i, samt hastighed mm.}
{Void}
{}