%!TEX root = ../main.tex

\chapter{Accepttest}

\begin{table}[H]
\centering
{\rowcolors{2}{white!80!black!30}{white!70!black!60} %farver på hver anden række -starter på 3
\setlength{\arrayrulewidth}{0.2mm}					 %tykkelse på linier 
\setlength{\tabcolsep}{10pt}						 %indryk i celle 
\renewcommand{\arraystretch}{1.5}					 %højden på tabelrum
\center
\begin{tabular}{|p{4cm}|p{4cm}|p{4cm}|}		 %længden på alle rum
\hline
\multicolumn{3}{|>{\columncolor{white!20!black!90}}m{13.44cm}|}{\textcolor{white}{\large{\textbf{Revision}}}} \\\hline
\rowcolor{white!70!black!60}
\textcolor{black}{\large{\textbf{Ændret af}}}&
\textcolor{black}{\large{\textbf{Version}}}&	
\textcolor{black}{\large{\textbf{Dato}}}\\
\hline
Alle	& 1	 	& 23-02-2015  \\
		& 		&   \\
		& 		&   \\
		& 	 	&   \\
\hline
\end{tabular}
}
\caption{Revision for accepttest}
\label{table:RevAccept}
\end{table}

\section{Test setup}
Til at teste følgende skal der bruges et setup med en PC der er i stand til at forbinde til den indlejrede Linux platform. Sensorer tilsluttes til den indlejrede Linux platform via de PSoC moduler der styre dem. Aktuatorerne skal ligeledes tilsluttes gennem deres respektive moduler.
  
\section{Accepttests}
%Accepttest til Usecase1
%!TEX root = ../main.tex

\begin{table}[H]
\centering
{\rowcolors{2}{white!80!black!30}{white!70!black!60} %farver på hver anden række -starter på 3
\setlength{\arrayrulewidth}{0.2mm}					 %tykkelse på linier 
\setlength{\tabcolsep}{10pt}						 %indryk i celle 
\renewcommand{\arraystretch}{1.5}					 %højden på tabelrum
\center
\begin{tabular}{ |p{2cm}|p{2cm}|p{2cm}|p{5.1cm}|}		 %længden på alle rum
\hline

\multicolumn{4}{|>{\columncolor{white!20!black!90}}m{13.27cm}|}{\textcolor{white}{\large{\textbf{Accepttest Use Case 1 Aflæs Data}}}} \\\hline
\rowcolor{white!70!black!60}
\textcolor{black}{\large{\textbf{Test}}}&
\textcolor{black}{\large{\textbf{Forventet resultat}}}&	
\textcolor{black}{\large{\textbf{Resultat}}}&
\textcolor{black}{\large{\textbf{Godkendt/Kommentar}}}\\
\hline
Test1	 	& Hat1	 	& Top 	& X - Det lort!\\
Test2  	 	& TopHat	& Hat 	& X - Mangler Top\\
Test3 	 	& KatHat 	& Kat 	& X - hvor er Hat\\
Test4   	& Hat2 		& Flaf 	& X - Flaf???\\
Test5 		& Top		& Mus 	& X - HUH?\\
Test6 		& Flaf 		& Hus   & X - No flaf\\
Test7 		& HatFlaf	& Tis 	& X - Ingen Tis\\
\hline
\end{tabular}
}
\caption{Accepttest 1}
\label{table:Atest1}
\end{table}

%Accepttest til Usecase2
%!TEX root = ../main.tex

\begin{table}[H]
\centering
{\rowcolors{2}{white!80!black!30}{white!70!black!60} %farver på hver anden række -starter på 3
\setlength{\arrayrulewidth}{0.2mm}					 %tykkelse på linier 
\setlength{\tabcolsep}{10pt}						 %indryk i celle 
\renewcommand{\arraystretch}{1.5}					 %højden på tabelrum
\center
\small
\begin{tabular}{|p{4cm}|p{4cm}|p{3cm}|p{2.5cm}|}		 %længden på alle rum
\hline

\multicolumn{4}{|>{\columncolor{white!20!black!90}}m{15.67cm}|}{\textcolor{white}{\large{\textbf{Accepttest Use Case 2 Indlæs Data}}}} \\\hline
\rowcolor{white!70!black!60}
\textcolor{black}{\large{\textbf{Test}}}&
\textcolor{black}{\large{\textbf{Forventet resultat}}}&	
\textcolor{black}{\large{\textbf{Resultat}}}&
\textcolor{black}{\large{\textbf{Godkendt Kommentar}}}\\
\hline
Bruger tilgår webinterfacet ved at indtaste url i en browser 	& Bruger indtaster MGMTURL	 	&  	& \\
Systemmet viser interface forside	 			& Visuel test forsiden fremkommer 	& 	&  \\
Bruger indtaster ønskede data i interfacet	 	& Fugtighed sætter til 34&  	& \\
Bruger indtaster ønskede data i interfacet	 	& pH-værdi sættes til 6,5&  	& \\
Bruger indtaster ønskede data i interfacet	 	& Gødning\#1 sættes til 90&  	& \\
Bruger indtaster ønskede data i interfacet	 	& Gødning\#2 sættes til 100&  	& \\
Bruger indtaster ønskede data i interfacet	 	& Gødning\#3 sættes til 110&  	& \\
Bruger gemmer data i systemmet  				& Bruger trykker på 'Gem værdier'	& 	& \\
Bruger lukker og åbner browser 					& Bruger lukker browseren og bruger åbner browseren igen & & \\
Bruger tilgår webinterfacet ved at indtaste url i en browser 	& Bruger indtaster MGMTURL	 	&  	& \\
Bruger aflæser værdierne 						& Brugeren sammenligner værdierne med de forige indtastede, disse skal være ens & & \\

\hline
\end{tabular}
}
\caption{Accepttest 2}
\label{table:Atest2}
\end{table}

%Accepttest til Usecase3
%!TEX root = ../main.tex

\begin{table}[H]
\centering
{\rowcolors{2}{white!80!black!30}{white!70!black!60} %farver på hver anden række -starter på 3
\setlength{\arrayrulewidth}{0.2mm}					 %tykkelse på linier 
\setlength{\tabcolsep}{10pt}						 %indryk i celle 
\renewcommand{\arraystretch}{1.5}					 %højden på tabelrum
\center
\begin{tabular}{ |p{2cm}|p{2cm}|p{2cm}|p{5.1cm}|}		 %længden på alle rum
\hline

\multicolumn{4}{|>{\columncolor{white!20!black!90}}m{13.27cm}|}{\textcolor{white}{\large{\textbf{Accepttest Use Case 3 Manuel vanding}}}} \\\hline
\rowcolor{white!70!black!60}
\textcolor{black}{\large{\textbf{Test}}}&
\textcolor{black}{\large{\textbf{Forventet resultat}}}&	
\textcolor{black}{\large{\textbf{Resultat}}}&
\textcolor{black}{\large{\textbf{Godkendt/Kommentar}}}\\
\hline
Test1	 	& Hat1	 	& Top 	& X - Det lort!\\
Test2  	 	& TopHat	& Hat 	& X - Mangler Top\\
Test3 	 	& KatHat 	& Kat 	& X - hvor er Hat\\
Test4   	& Hat2 		& Flaf 	& X - Flaf???\\
Test5 		& Top		& Mus 	& X - HUH?\\
Test6 		& Flaf 		& Hus   & X - No flaf\\
Test7 		& HatFlaf	& Tis 	& X - Ingen Tis\\
\hline
\end{tabular}
}
\caption{Accepttest 3}
\label{table:Atest3}
\end{table}

%Accepttest til Usecase4
%!TEX root = ../main.tex

\accepttest[UC3]{Use Case 4 Karstyring}{
Tryk på Karstyring på interfacet	& Der forekommer 2 valmuligheder &  & \\
Tryk på PH-værdi   					& Der gives mulighed for at indtaste data	& 	& \\
Indtast 7 og tryk ok 				& PH-værdien opdateres til 7   &	& \\
Tryk på OK   						& Menuen returnerer   &  	& \\
Tryk på Volumen 					& Der gives mulighed for at indtaste data		& 		& \\
Indtast 100 						& Volumen opdateres til 100L 		&    & \\
Tryk på OK 		    				& Menuen returnerer	        &    & \\
Tryk på OK 							& Cirkulations pumpe og pumperne til dosering af gødningen starter	&	  	& \\
}