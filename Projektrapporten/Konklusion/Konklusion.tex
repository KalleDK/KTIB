%!TEX root = ../main.tex
\chapter{Konklusion}
Fra projektes start var det ikke meningen at projektet skulle fylde for meget i forhold til undervisningen. Det var derfor hensigten af finde en tilpas arbejdsbyrde hvor hver af gruppens medlemmer kunne nå at både følge med i det daglige og samtidig have tid til projektet. Det viste sig hurtigt at systemet havde vokset sig meget stort og der blev derfor en skævvridning af arbejdsbyrden. Kravende til hardwaren blev hurtigt lavet og det var derfor nemmere for hardware folkene at danne sig et overblik over hvad der skulle laves. Softwaren blev ændret meget over tiden og da der hele tiden blev fundet nye og bedre måder at løse problemstillingen på, krævede det en del tid at få ændringerne implementeret. Undervejs har vi haft stor tillid til hinanden og har derfor ikke haft en direkte styringsmetode. Vi har prøvet at køre noget ligene scrum hvor vi har haft perioder hvor vi har aftalt at køre et intenst arbejdsforløb. Dette har vist sig at fungere godt på trods af at det ikke har fungeret optimalt. Skulle det have fungeret optimalt, skulle der have været en skrappere styring af fordelingen af arbejdet og der skulle have været en bedre information indbyrdes.   