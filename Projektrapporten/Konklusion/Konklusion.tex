%!TEX root = ../main.tex
\chapter{Konklusion}
I forhold til den stillede opgave og efterfølgende afgrænsning er der blevet lavet en prototype der virker meget godt. Dog var der til start større ambitioner der blev aldrig implementeret. Systemet skulle kunne dosere gødning til vandet i karret for at give optimale vækstforhold et sådant modul blev ikke implementeret. Dette var grundet tidspres og tekniske problemer, som nødvendiggjorde at fokus blev lagt på den basale funktionalitet. \\ Ændringen i fokus gjorde så at vi kunne lave prototypen som kom til at opfylde de krav der var sat til første iteration. Heri ligger også at mange af de automatiske funktioner ikke blev implementeret. \\
Selve projekt styringen har også haft lidt problemer. Det blev forsøgt styret via Scrum, som er et godt værktøj til at arbejde i iterationer. Der var dog udfordringer i forhold til hvem der skulle være produkt owner og hvem der skulle være Scrum master. \\
Rollen som Scrum master kørte meget løst og det havde helt sikkert været et plus hvis denne havde været bedre defineret. Scrum bordet virkede fint på den måde vi brugte det gennem waffle.io, hertil var dog at vi kun havde en sprint backlog og ikke en produkt backlog.\\
De daglige sprint møder fungerede ganske godt, dog kunne det godt mærkes at vi ikke eksklusivt lavede projekt da det ikke var hverdag at der var blevet gennemført opgaver. \\
Til at dokumentere projektet virkede SysML meget godt fordi det kan give et super overblik. Der manglede dog en smule koordinering mellem de forskellige hardware/software enheder, dette kunne være afhjulpet med en bedre styring af projektet gennem bedre brug af scrumm.\\\\\\\\\\




Fra projektes start var det ikke meningen at projektet skulle fylde for meget i forhold til undervisningen. Det var derfor hensigten af finde en tilpas arbejdsbyrde hvor hver af gruppens medlemmer kunne nå at både følge med i det daglige og samtidig have tid til projektet. Det viste sig hurtigt at systemet havde vokset sig meget stort og der blev derfor en skævvridning af arbejdsbyrden. Kravende til hardwaren blev hurtigt lavet og det var derfor nemmere for hardware folkene at danne sig et overblik over hvad der skulle laves. Softwaren blev ændret meget over tiden og da der hele tiden blev fundet nye og bedre måder at løse problemstillingen på, krævede det en del tid at få ændringerne implementeret. Undervejs har vi haft stor tillid til hinanden og har derfor ikke haft en direkte styringsmetode. Vi har prøvet at køre noget ligene scrum hvor vi har haft perioder hvor vi har aftalt at køre et intenst arbejdsforløb. Dette har vist sig at fungere godt på trods af at det ikke har fungeret optimalt. Skulle det have fungeret optimalt, skulle der have været en skrappere styring af fordelingen af arbejdet og der skulle have været en bedre information indbyrdes.   