%!TEX root = ../main.tex
\chapter{Konklusion}
I forhold til den stillede opgave og efterfølgende afgrænsning er der blevet lavet en prototype, der virker meget godt. Dog var der til start større ambitioner, som blev aldrig implementeret. Systemet skulle kunne dosere gødning til vandet i karret for at give optimale vækstforhold. Et sådant modul blev ikke implementeret. Dette var grundet tidspres og løbende tekniske problemer, som nødvendiggjorde, at fokus blev lagt på den basale funktionalitet. \\ Ændringen i fokus gjorde så, at vi kunne lave prototypen, som kom til at opfylde de krav, der var sat til første iteration. Heri ligger også, at mange af de automatiske funktioner ikke blev implementeret.\\\\
Selve projektstyringen har også haft lidt problemer. Det blev forsøgt styret via Scrum, som er et godt værktøj til at arbejde i iterationer. Der var dog udfordringer i forhold til hvem der skulle være produkt owner og hvem der skulle være Scrum master. Rollen som Scrum master kørte meget løst og det havde helt sikkert været et plus hvis denne havde været bedre defineret. Scrum bordet virkede fint på den måde vi brugte det gennem waffle.io, hertil var dog, at vi kun havde en sprint backlog og ikke en produkt backlog.\\\\
De daglige sprintmøder fungerede ganske godt, dog kunne det godt mærkes, at vi ikke eksklusivt lavede projekt, da det ikke var hver dag der var blevet gennemført opgaver, og fokus også lå andre stedet i studiet.\\\\
Til at dokumentere projektet virkede SysML meget godt fordi det kan give et super overblik. Der manglede dog en smule koordinering mellem de forskellige hardware/software enheder, dette kunne være afhjulpet med en bedre styring af projektet gennem bedre brug af scrumm.
