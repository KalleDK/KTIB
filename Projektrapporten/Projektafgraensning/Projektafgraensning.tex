%!TEX root = ../main.tex
\chapter{Projektafgrænsning}
Til dette projekt har vi fra start af haft nogle hardwarespecifikke krav, der siger, at vi skulle bruge sensorer/aktuatorer, systemet skulle have en grafisk brugergrænseflade og systemet skulle indeholde en indlejret Linux platform og en PSoC platform. Ud fra disse krav lavede vi en opgaveformulering som vi følte dækkede kravene. \\
I denne formulering skulle systemet være autonomt og selv være i stand til at vande, doserer gødning i vandet, for at give optimale vækst forhold, og opsamle data hvorfra systemet kunne tage sine beslutninger. Der skulle også være en høj grad af automation i forhold til sensorer og deres tilkobling til systemet. Disse skulle kunne autodetekteres således, at brugerne ikke skulle skænke det en tanke under operation. Systemet skulle også kunne understøtte flere kar og sensor ø'er. \\
Vi valgte dog ikke at fokusere på alt automatikken, da der var andre udfordringer der kom på banen, i form af bl.a. store afstande mellem delsystemerne, muligheden for konstruktion af egne sensorer, herunder disses kommunikation, og PSoC-platformens mulighed for modulopdeling igennem specialfremstillede komponenter. \\
Derfor valgte vi at fokusere på et system med basal funktionalitet med mulighed for senere udvidelse såfremt tiden tillod det. Dog var afstandene i systemet ikke noget vi ønskede at gå på kompromis med. Den basale funktionalitet kommer til udtryk i, at systemet kan vande og opsamle målte data fra sensorer. Der er ligeledes implementeret en brugergrænseflade, som er nem at udvide med flere kar. Kommunikationen er også lavet efter muligheden for udvidelser. \\\\\\

