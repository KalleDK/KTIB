%!TEX root = ../../main.tex
\section{Metoder}
\subsection{Scrumm}
Til styring af arbejdsprocessen i dette projekt har vi forsøgt at bruge scrumm, det har vi forsøgt at udføre igennem iterationer af projektet samt fjorten dages sprint hvor der så vidt som muligt har været holdt daglige møder. Dog har vi ikke igennem hele projekt perioden brugt scrumm der har været perioder hvor det ikke har været hensigtsmæssigt som i krav specifikation og system arkitektur fasen af projektet. I vores brug af scrumm har der heller ikke været en "produkt owner" det har været gruppen der har skulle styre backloggen og sprintbackloggen, der har heller ikke været en decideret scrumm master. Det er altså arbejdsgangen vi har benytte blandt andet igennem brug af scrumm board via waffle.io som integrerer med vores projekt på Github. 

\subsection{ASE modellen}
Til at udforme dokumentationen er ASE modellen blevet benyttet, det vil sige at projektet er delt op i nogle faser hvor hver enkelt fase producere et dokument. Der har været lidt udfordring i forhold til at få dette til at køre sammen med scrumm derfor har design fasen som nævnt ikke kørt efter scrumm.

\subsection{SysML}
Til udformning af krav specifikation og system arkitektur er der brugt SysML både til at udforme use cases som beskriver de funktionelle krav for systemet og senere blok definitions diagrammer (BDD) og interne blok diagrammer (IDB) der er brugt til at give et indblik i hardware opbygningen. Til den overordnede beskrivelse af systemet er der brugt en domæne model diagram som også er blevet brugt til at udforme en applikations model som forsøger at beskriver hvordan software hænger sammen her er der også brugt system sekvens diagrammer.

\subsection{•}