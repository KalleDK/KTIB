%!TEX root = ../../main.tex
\section{Resultater og Diskussion}
I bund og grund opnåede vi meget af det vi ville i forhold til en første iteration/prototype. Der blev lavet en grafisk brugergrænseflade, der kører godt, som kan det den skal i forhold til de krav der er sat. FlexPMS kører godt og har den basale funktionalitet, men her er plads udvidelser, blandt andet i forhold til automatisering.\\
Der var en del besvær i forhold til kommunikationen mellem FlexPMS og Kar. RS485-bussen blev udvidet med en selvudviklet protokol, der indeholdte addresering. I denne sammenhæng var adresseringen et problem, da vi brugte paritetsbit'en i UART data-frames til at indikere adresser. Dog var der ingen understøttelse for dette på Linux-platformen. Her kunne protokollen forbedres og implementeringen kan sagtens optimeres.\\
Kar- og Ø-styringerne har også den basale funktionalitet på plads. Ø'en kan skanne og finde fieldsensorer, så her er opnået lidt af den automatisering, der skulle implementeres.\\
I forhold til hardwaren er der blevet lavet nogle gode udlæg til print for jordfugtsensoreren. Der var desværre en fejl i de udlagte print men prototypen virkede efter hensigten. \\
Ventilstyringen fungere også ganske godt, dog var der lidt problemer i forhold til støj fra pumpe-styringen.\\
Strømforsyningen virker ganske godt og kan levere det, som den skal. \\
Flowsensoren virkede også, men der var dog en udfordring i forhold til at kunne måle rate of flow, da PSoC'en ikke har et begreb omkring tid.\\
Der er også blevet designet print til alle PSoCs samt Raspberry pi. Vi har dog ikke nået at implementere dem til denne rapports aflevering.
   

\subsection{Støjproblemer i prototype-setup}

Ved realisering af system-prototypen til styring af indløb- og afløbsventiler samt doseringspumpe blev styringskredsløb til disse implementeret på samme veroboard. 
Dette gav nogle uforudsete støjproblemer, idet pumpen (når denne kørte på 100\% effekt, og trak 2.2A) støjede så meget at der lå nok spænding over styringen af ventilerne til at få dem til at gå "on" og "off".\newline

Problemet synes at kunne komme flere steder fra: 
Dels at designet er implementeret på veroboard og afstanden imellem de fortrykte copperbaner kan give anledning til støjoverlejring
Dels at Kar-PSoC'en ved teststillingen var en "fuglerede" af tilslutninger. 
Dels var ventilerne "kun" er afkoblede med 100nF caps, og dette sikre dem desværre ikke imod denne type støj.\newline

Der blev prøvet flere løsninger til dette problemet: 
Fysisk blev printet med doseringspumpe-styringen adskilt adskilt fra ventilstyringerne, så muligheden for overlejring her blev elimineret. 
Derudover blev det forsøgt at skifte PSoC'ens output ben til pumpestyringen. Dette havde desværre heller ikke den ønskede effekt. \newline

Det var også på tale om støjen evt. kunne løbe tilbage igennem strømforsyningen, og dermed forstyrre hele 12V linjen. 
Da vi ikke fandt noget endegyldigt svar på problemet. Blev løsningen at nedjustere doseringspumpens "100\%" således at den ved max. kun trækker 1,8A, dette var rigeligt til formålet, og samtidig gav det ikke problemer med nogle af de andre tilkoblede kredsløb.
\newline