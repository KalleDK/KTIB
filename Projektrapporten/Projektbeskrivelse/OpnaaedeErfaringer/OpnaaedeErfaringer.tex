%!TEX root = ../../main.tex
\section{Opnåede Erfaringer}
I opnåede erfaringer vil vi hver især reflektere på projektets forløb og hvad vi kan tage med derfra. De opnåede erfaringer i projektet vurderes individuelt, da hver projektdeltager har oplevet det på sin egen måde. Der kan forekomme ens opnåede erfaringer, men hvor lærerigt det har været kan være forskelligt fra projektdeltager til projektdeltager.

\subsection{Jakob}
Jeg har udviklet en stor del af FlexPMS, og har i den forbindelse benyttet mig af mange af de metoder, som vi har lært i ISU. Jeg har haft specielt fokus på, at få implementeret de metoder vi har lært omkring eventbaseret tråd-kommunikation, da det var oplagt til produktet, og har desuden suppleret det med egne undersøgelser af TCP/IP socket-forbindelser og MySQL-forbindelse i C++. I sidstnævnte forbindelse har jeg opnået en god erfaring med kompilering og linking af 3. parts biblioteker i C++. Da FlexPMS er et relativt stort program (i forhold til vores tid og tidligere erfaringer) var det også en udfordring at få designet programmet, så boundaryklasser, domæneklasser og controllers var pænt adskilt men alligevel kunne interagere med hinanden på fornuftig vis. Her har især sekvensdiagrammer været en stor hjælp.\\
I løbet af projektet har jeg desuden fået god indsigt i, hvordan Git fungerer, en disciplin som jeg kun havde svag erfaring med før vi startede, men som jeg ikke er i tvivl om er en rigtig god kompetence at besidde.\\
Mit arbejde på FlexPMS har foregået i tæt samarbejde med Kasper, som lavede alt kommunikation ud til de fysiske busser. Det har været et godt forløb, hvor vi har suppleret hinanden løbende, og delt de erfaringer vi har opnået gennem især ISU øvelserne.

\subsection{Kalle}
I dette projekt har jeg arbejdet mest med udvikling af PSoC enhederne Kar, SensorØ og Fieldsensor. Desuden har jeg  stået for en del af de bagvedlæggende systemer til vores "udviklingsmiljø" så som github, waffle.io, sharelatex og travis. Jeg har brugt en del viden fra GFV og nogle design patterns fra ISU. Den mest udfordrende del kommer helt sikkert fra AVSLibraryet, da vi ikke har fået undervisning i custom komponenter, samt PSoC Creator har store forskelle mellem de forskellige versioner. Det har dog været ret tilfredsstillende at bagefter kunne lave drag'n'drop med ens egne komponenter, da det har giver meget større overskuelighed på de enkelte enheder, samt været utrolig let at rette fejl.\\\\
Ved at stå for de bagvedlæggende systemer har jeg fået meget erfaring i at koble systemerne sammen, og derved letne arbejdsbyrden, jeg nåede ikke at få mig sat ind i Doxygen til dokumentation, og dette er helt sikkert en fejl i forhold til tid der er blevet brugt på dokumentation af kode. Der har desuden også været lidt mangler i styring af projektet. 

\subsection{Karsten}
I min del af projektet er der blevet arbejdet med pH-proben og jordfugtmåleren. I arbejdet er der blevet inkluderet viden fra fagene EFYS, GFV og MSE. Det har været spændende at inkludere viden fra sideløbende undervisning og det har på den måde fremmet forståelsen for fagene. Undervejs har det været en udfordring at vide hvad der skete på softwaresiden, da systemerne hele tiden blev lavet om og meget af det der er blevet lavet er uden for mit vidensområde.
\\\\
Af opnåede erfaringer kan nævnes udlægning af print som har været helt nyt. Her har der været nye programmer der skulle læres at kende og det har givet et godt grundlag for efterfølgende semesterprojekter, da dette vil lette arbejdet med prototyperne. Endvidere er projektet blevet lavet via et revisionsstyringssystem samt rapporten er blevet skrevet i LaTeX. Dette har vist sig at være særdeles nyttigt. 

\subsection{Kasper}
Til at starte med har jeg undervurderet hvor hurtigt dette projekt forløb kom til at gå, der har været fuld fart på fra start af. Når et forløb går så stærkt er det vigtigt at have en leder der har det store overblik, vi har kørt det mere løst hvor der ikke var en der stod for at samle os og følge op på arbejde, det er noget jeg vil tage med videre. Dog føler jeg at det er svært at når projekt arbejdet kører sideløbende med undervisningen i andre fag. Min egen rolle i dette projekt har været software udvikling og her har jeg lært at en god plan kan gøre meget for at lette arbejdet. Hvis der ikke er en aftalt plan så giver det problemer hen af vejen. Det har været sjovt at kæde de forskellige 3. semesters fag ind i projektet det har også givet anledning til ændringer undervejs når man lige lærte et nyt design pattern i ISU, eller fandt ud af at PSoC platformen kunne noget mere i GFV. \\
Alt i alt har det været et udfordrende og læringsrigt projekt som har givet meget i forhold til hvad der skal til for at styre et projekt, også i forhold til hvis der kører agile metoder. Det er vigtigt at have en leder om det så er i form af en Scrum Master eller andet er mindre vigtig. det har også givet meget i forhold til at bruge det jeg har lært i undervisningen. Det har også været lækkert at udvide min kendskab til værktøj som git og LaTeX.

\subsection{Kenn}
I dette projekt har jeg primært arbejdet med på hardware-siden, med design og implementering af styringskredsløb til ventiler og pumpe, samt counter-kredsløb til flowsensoren. Derudover har jeg udviklet første iteration af software, samt tests til ventil- og pumpestyring. Tilsidst har jeg stået for at designe og udlægge prints til de 3 shields, hvilken grundet tidspres og leverngsdatoer ikke nåede at komme 100\% med i dokumentationen, men disse forventes at være klar til præsentation til fremlæggelsen. I mit arbejde har jeg trukket på den erfaring jeg har fået, primært i E3MSE, E3FYS samt GFV. MSE og GFV primært til udarbejdelse af styringskredsløb, og EFYS i forbindelse med udlægning af print til forståelse af støjoverlejringer.\newline
Af opnåede erfaring har det været helt nyt for mig at arbejde med Eagle og udlægning af print med alt hvad det indebar. Dette er helt sikkert en fordel at ha lært, både til senere samesterprojekter, samt til brug udenfor skolen. Derudover er der i projektet blevet benyttet LaTeX til fremstilling af dokumenter, samt git til versioncontrol, begge disse 2 værktøjer har krævet en del tid at lærer at administrere. Til sidst har det, i dette projekt, til forskel fra de tidligere semestre været meget mere adskilt SW/HW. Derfor har det været utrolig lærerrigt at læse softwarearkitekturen som "udefrakommende", og forstå denne.

\subsection{Lærke}
I arbejdet med projektet har jeg opnået en lang række faglige erfaringer. Jeg arbejdede med GUI'en og derfor kom jeg ind på både php, html, databaser, jQuery, mm., hvilket har været uden for mit pensum og til trods for at skulle arbejde med et nyt sprog som f.eks. php, kunne jeg stadig bruge mine erfaringer med objekt-orienteret programmering. Men selvom der forekom problemer med at arbejde med noget uden for pensum, viste det sig at være nogle værktøjer, der løste mange af problematikkerne i kravspecifikationen.
\\\\
Da projektet er delt op i flere moduler, men også delt op i hardware og software var der en del overvejelser der skulle gennemtænkes. Dette demonstrerede hvor vigtigt det er at have en god kommunikation i gruppen, som også har vist, at det er en god ide at bruge de scrum inspirerede metoder til projekt gennemførelsen, da det altid var vigtigt at holde status på processen, men også kommunikationen mellem moduler. 

\subsection{Thomas}
I 3. semesterprojektet har jeg udover projektdokumentation arbejdet med PSU og RSConverter. I mit arbejde med hardware udvikling har jeg inddraget viden fra fagene EFYS, EEV, MSE og GFV. Strømforsyningen har været den største men bestemt også den mest interessante opgave jeg har arbejdet med. Jeg har haft mulighed for at udforske emner på egen hånd, som f.eks. regulering. Reguleringskredsløb er bygget op del for del, hvilket har medført mange ændringer undervejs, men også en god forståelse for hvad de forskellige fænomener skyldes.
\\\\
Projektet vi har valgt har været rigtig spændende. Forløbet fra idéen frem til prototypen har været meget lærerig. Den allervigtigste erfaring jeg tager med mig fra projektet er, hvor vigtigt det er med kommunikation imellem projektdeltagerne. En god kommunikation forhindrer misforståelser som senere hen kan skabe små såvel som store problemer.