%!TEX root = ../../main.tex
\section{Opnåede Erfaringer}
I opnåede erfaringer vil vi hver især reflektere på projektets forløb og hvad vi kan tage med derfra. De opnåede erfaringer i projektet vurderes invidividuelt, da hver projektdeltager har oplevet det på sin egen måde. Der kan forekomme ens opnåede erfaringer, men hvor lærerigt det har været kan være forskelligt fra projektdeltager til projektdeltager.

\subsection{Jakob}

\subsection{Kalle}
I dette projekt har jeg arbejdet mest med udvikling af PSoC enhederne Kar, SensorØ og Fieldsensor. Desuden har jeg  stået for en del af de bagvedlæggende systemer til vores "udviklingsmiljø" så som github, waffle.io, sharelatex og travis. Jeg har brugt en del viden fra GFV og nogle design patterns fra ISU. Den mest udfordrende del kommer helt sikkert fra AVSLibraryet, da vi ikke har fået undervisning i custom komponenter, samt PSoC Creator har store forskelle mellem de forskellige versioner. Det har dog været ret tilfredsstillende at bagefter kunne lave drag'n'drop med ens egne komponenter, da det har giver meget større overskuelighed på de enkelte enheder, samt været utrolig let at rette fejl.\\\\
Ved at stå for de bagvedlæggende systemer har jeg fået meget erfaring i at koble systemerne sammen, og derved letne arbejdsbyrden, jeg nåede ikke at få mig sat ind i Doxygen til dokumentation, og dette er helt sikkert en fejl i forhold til tid der er blevet brugt på dokumentation af kode. Der har desuden også været lidt mangler i styring af projektet. 

\subsection{Karsten}
I min del af projektet er der blevet arbejdet med pH-proben og jordfugtmåleren. I arbejdet er der blevet inkluderet viden fra fagene EFYS, GFV og MSE. Det har været spændende at inkludere viden fra sideløbende undervisning og det har på den måde fremmet forståelsen for fagene. Undervejs har det været en udfordring at vide hvad der skete på softwaresiden, da systemerne hele tiden blev lavet om og meget af det der er blevet lavet er uden for mit vidensområde.
\\\\
Af opnåede erfaringer kan nævnes udlægning af print som har været helt nyt. Her har der været nye programmer der skulle læres at kende og det har givet et godt grundlag for efterfølgende semesterprojekter, da dette vil lette arbejdet med prototyperne. Endvidre er projektet blevet lavet via et revisionsstyringssystem samt rapporten er blevet skrevet i LaTeX. Dette har vist sig at være særdeles nyttigt. 
\subsection{Kasper}

\subsection{Kenn}

\subsection{Lærke}

\subsection{Thomas}
I 3. semesterprojektet har jeg udover projektdokumentation arbejdet med PSU og RSConverter. I mit arbejde med hardware udvikling har jeg inddraget viden fra fagene EFYS, EEV, MSE og GFV. Strømforsyningen har været den største men bestemt også den mest interessante opgave jeg har arbejdet med. Jeg har haft mulighed for at udforske emner på egen hånd, som f.eks. regulering. Reguleringskredsløb er bygget op del for del, hvilket har medført mange ændringer undervejs, men også en god forståelse for hvad de forskellige fænomener skyldes.
\\\\
Projektet vi har valgt har været rigtig spændende. Forløbet fra idéen frem til prototypen har været meget lærerig. Den allervigtigste erfaring jeg tager med mig fra projektet er, hvor vigtigt det er med kommunikation imellem projektdeltagerne. En god kommunikation forhindrer misforståelser som senere hen kan skabe små såvel som store problemer.